\begin{examnotes}{Exam 1 Notes}
    \subsubsection*{Data Frames}

    A data frame in Pandas is a two-dimensional, size-mutable, and potentially heterogeneous tabular data structure with labeled axes (rows and columns). It's akin to a spreadsheet or SQL table and 
    is one of the most commonly used Pandas data structures.

    You can create a data frame from various sources, such as:

    \begin{itemize}
        \item Lists
        \item Dictionaries
        \item NumPy arrays
        \item CSV files
        \item SQL databases
    \end{itemize}

    The structure of data frames can be summed up by:

    \begin{itemize}
        \item \textbf{Rows}: Each row represents a single observation or record.
        \item \textbf{Columns}: Each column represents a variable or feature. Columns can be of different data types (integer, string, float, etc.).
        \item \textbf{Index}: This is the `key' for rows, similar to an index in a database. It’s an immutable array, allowing fast access to data.
    \end{itemize}

    Some operations that can be used with data frames are:

    \begin{itemize}
        \item \textbf{Data Manipulation}: Adding, deleting, and modifying both rows and columns.
        \item \textbf{Filtering}: Selecting a subset of rows or columns based on some criteria.
        \item \textbf{Sorting and Grouping}: Organizing data based on values in certain columns.
        \item \textbf{Merging and Joining}: Combining multiple data frames.
        \item \textbf{Handling Missing Data}: Identifying and imputing missing values.
    \end{itemize}

    \begin{highlight}[Data Frame Operation Examples]
        Here are some examples of data frame manipulation in pandas:

        \subsubsection*{Creation}

    \begin{code}[Python]
    import pandas as pd

    # Creating a data frame from a dictionary
    data = {'Name': ['Alice', 'Bob', 'Charlie'], 
            'Age': [25, 30, 35],
            'City': ['New York', 'Los Angeles', 'Chicago']}
    df = pd.DataFrame(data)            
    \end{code}

        \subsubsection*{Adding A Column}

    \begin{code}[Python]
    # Adding a new column
    df['Salary'] = [70000, 80000, 90000]        
    \end{code}

        \subsubsection*{Deleting A Column}

    \begin{code}[Python]
    # Deleting a column
    df.drop('Age', axis=1, inplace=True)        
    \end{code}

        \subsubsection*{Filtering Data}

    \begin{code}[Python]
    # Filtering rows where Salary is greater than 75000
    high_earners = df[df['Salary'] > 75000]        
    \end{code}

        \subsubsection*{Sorting Data}

    \begin{code}[Python]
    # Sorting data by Salary in descending order
    df_sorted = df.sort_values(by='Salary', ascending=False)        
    \end{code}

        \subsubsection*{Merging Data Frames}

    \begin{code}[Python]
    # Creating another data frame
    additional_data = pd.DataFrame({'Name': ['Alice', 'Bob'], 'Experience': [5, 10]})
    
    # Merging data frames
    merged_df = pd.merge(df, additional_data, on='Name', how='left')        
    \end{code}

        \subsubsection*{Handling Missing Data}

    \begin{code}[Python]
    # Filling missing values with zero
    df_filled = df.fillna(0)        
    \end{code}

        \subsubsection*{Reading Data}

    \begin{code}[Python]
    # Reading data from a CSV file
    df_from_csv = pd.read_csv('data.csv')
    
    # Writing data to a CSV file
    df.to_csv('output.csv', index=False)        
    \end{code}
    \end{highlight}

    \subsubsection*{Combinatorics}

    Combinatorics is a branch of mathematics dealing with the study of countable, discrete structures and their properties. It's particularly important in computer science, where understanding how to 
    count and arrange objects is crucial for algorithm design, data structure optimization, and problem-solving. Here's a summary of the key concepts in combinatorics:

    \begin{itemize}
        \item \textbf{Counting Principles}
        \begin{itemize}
            \item \textbf{The Rule of Sum}: If there are $A$ ways to do something and $B$ ways to do another thing, and these two things cannot happen at the same time, then there are $A + B$ ways 
            to choose one of these actions.
            \item \textbf{The Rule of Product}: If there are $A$ ways to do something and $B$ ways to do another thing after that, then there are $A \cdot B$ ways to perform both actions.
        \end{itemize}
        \item \textbf{Permutations}
        \begin{itemize}
            \item Permutations are the arrangements of objects in a specific order.
            \item The number of permutations of $n$ distinct objects is $n!$ ($n$ factorial), which is the product of all positive integers up to $n$.
            \item For arranging $r$ objects out of $n$ available objects, the formula is
            \begin{equation*}
                ^{n}P_{r} = \frac{n!}{(n - r)!}.
            \end{equation*}
        \end{itemize}
        \item \textbf{Combinations}
        \begin{itemize}
            \item Combinations refer to the selection of objects without regard to the order.
            \item The number of ways to choose $r$ objects from $n$ different objects is given by 
            \begin{equation*}
                \begin{pmatrix}
                    n \\
                    r \\
                \end{pmatrix}
                = \frac{n!}{r!(n - r)!}.
            \end{equation*}
            \item Combinations are used when the order doesn't matter.
        \end{itemize}
        \item \textbf{Binomial Theorem}
        \begin{itemize}
            \item It provides a formula for the expansion of powers of a binomial (sum of two terms).
            \item The Binomial Theorem states that:
            \begin{equation*}
                (a + b)^{n} = \sum_{k = 0}^{n} 
                \begin{pmatrix}
                    n \\
                    k \\
                \end{pmatrix}
                \cdot a^{n - k} \cdot b^{k}
            \end{equation*}
            \begin{itemize}
                \item This means the expansion is a sum of terms, where the exponents of a start at $n$ and decrease to 0, while the exponents of $b$ start at 0 and increase to $n$. The coefficients 
                of each term are the corresponding binomial coefficients.
            \end{itemize}
            \item The coefficients of the terms in the expansion are the binomial coefficients, which can be calculated using combinations.
        \end{itemize}
        \item \textbf{Binomial Distribution}
        \begin{itemize}
            \item A binomial distribution is a discrete probability distribution that models the number of successes in a fixed number of independent Bernoulli trials. A Bernoulli trial is an experiment 
            with exactly two possible outcomes, typically termed "success" and "failure".
            \item In the context of the binomial distribution, $P(x = k)$ denotes the probability of getting exactly $k$ successes in $n$ trials. The formula for this is
            \begin{equation*}
                P(x = k) = 
                \begin{pmatrix}
                    n \\
                    k \\
                \end{pmatrix}
                \cdot p^{k} \cdot (1 - p)^{n - k}.
            \end{equation*}
            \begin{itemize}
                \item $\binom{n}{k}$ (read as "$n$ choose $k$") is the binomial coefficient, representing the number of ways to choose $k$ successes from $n$ trials.
                \item $p$ is the probability of success on an individual trial.
                \item $1 - p$ is the probability of failure (since the trials are binary, the sum of the probabilities of success and failure is 1).
                \item $p^{k}$ is the probability of having $k$ successes.
                \item $(1 - p)^{n - k}$ is the probability of having $n - k$ failures.
            \end{itemize}
        \end{itemize}
    \end{itemize}
\end{examnotes}